\documentclass[a4paper,10pt]{report}
\usepackage[T1,T2A]{fontenc}
\usepackage[utf8]{inputenc}
\usepackage{lmodern}
\usepackage{minted}
\usepackage[colorlinks=true, linkcolor=blue, urlcolor=blue]{hyperref}
\usepackage[cm]{fullpage}

\makeatletter
\newcommand{\minted@write@detok}[1]{%
  \immediate\write\FV@OutFile{\detokenize{#1}}}%

\newcommand{\minted@FVB@VerbatimOut}[1]{%
  \@bsphack
  \begingroup
    \FV@UseKeyValues
    \FV@DefineWhiteSpace
    \def\FV@Space{\space}%
    \FV@DefineTabOut
    %\def\FV@ProcessLine{\immediate\write\FV@OutFile}% %Old, non-Unicode version
    \let\FV@ProcessLine\minted@write@detok %Patch for Unicode
    \immediate\openout\FV@OutFile #1\relax
    \let\FV@FontScanPrep\relax
%% DG/SR modification begin - May. 18, 1998 (to avoid problems with ligatures)
    \let\@noligs\relax
%% DG/SR modification end
    \FV@Scan}
    \let\FVB@VerbatimOut\minted@FVB@VerbatimOut

\renewcommand\minted@savecode[1]{
  \immediate\openout\minted@code\jobname.pyg
  \immediate\write\minted@code{\expandafter\detokenize\expandafter{#1}}%
  \immediate\closeout\minted@code}
\makeatother

\title{Yandex.Root Final}
\author{Igor Gnatenko (Linuxorgru)}

\begin{document}

\maketitle
\tableofcontents

\begin{abstract}
\end{abstract}

\part{Приготовления и запуск}

2 образа. Debian, OpenIndiana. Всё как обычно, меняем пароль руту, грузимся.

\part{Задачи}

\chapter{Backup}
Our old admin supported small site with some very useful texts. Unfortunately, he quit some time ago and we do not know how to restore the site. Please make it run again with the most recent data available.
\newline\newline
\textbf{true\_admin}:
\newline
Сразу смотрим ~/.bash\_history пока его не затёрли (HISTSIZE=10!).
\newline
Оттуда понятно что бэкапы лежат в /var/backups и так же виден ключ и то что шифровалось через openssl.
Попутно выясняется что ключ подходит только к первому архиву. А тааам... ключ ко второму архиву в котором...
ну вы поняли.
\newline
\newline
\begin{minted}[fontsize=\scriptsize, samepage=true]{bash}
HISTSIZE=10000 иии...
cd /var/backups
KFILE=backup.key
TMPFILE=/tmp/temp.arch
# самый первый ключ из .bash_history
echo 27CBB6C8170EF54C3235AF3F8ED2A106 > $KFILE

# небольшой уикл
for f in `ls *encrypted | sort -r`; do
  newname=`basename $f .encrypted`;
  openssl aes-128-cbc -d -kfile $KFILE -in $f -out $newname;
  tar --strip-components=1 -xf $newname root/backup.key
done

for f in `ls *tgz | sort` ; do tar -xf $f; done
# копируем usr/share/nginx/www в корень.
mv usr/ /
\end{minted}

\chapter{HTTPS MITM}
We continue with our Internel filtration topic. Now you need to set up HTTP proxy on port 3129 which will intercept all HTTPS traffic coming through and do it properly: we expect the proxied sites to provide valid certificates for their associated host-names. You may sign it with our own MITM certificate authority (the CA key and certificate are available in /root), we will use a client that trusts this CA.
\newline
To additionally prove that you are ready to intercept traffic, please substitute the value for Server: header of all requests with the string "root.yandex.com".
\newline\newline
Ставим mitmproxy, запускаем и останавливаем для того чтобы сгенерировались сертификаты
\newline
потом \textit{cd /root ; cat mitm.key mitm.crt > .mitmproxy/mitmproxy-ca.pem}
\newline
Затем:
\newline
\begin{listing}[H]
  \begin{minted}[linenos=true, fontsize=\scriptsize, samepage=true]{python}
  def response(context, flow):
      flow.response.headers["Server"] = ["root.yandex.com"]
  \end{minted}
  \caption{add\_header.py}
  \label{lst:add_header.py}
\end{listing}
\begin{minted}[fontsize=\scriptsize, samepage=true]{text}
mitmproxy -p 3129 -s add_header.py
\end{minted}
Жмём | и запускаем game

\chapter{CI}
We got a very old continious integration system (on OpenIndiana) and we want to upgrade it.
\newline
Please upgrade project serverMVC to use Docker.
\newline
Out checker works with your repo at ssh://root@\$\{YOUR\_IP\}/root/app.git and for start deploy checker will use url: http://\$\{YOUR\_IP\}:8080/hudson/job/serverMVC\_docker/build?delay=0sec
\newline
Moreover, checker expects Docker running on \$\{YOUR\_IP\}.
\newline
Our id\_rsa.pub:
\begin{tiny}
ssh-rsa AAAAB3NzaC1yc2EAAAADAQABAAABAQDoGfF7Dvs5H0aeMsfm9MMasUWY12rdphM410FJdJ
aoAjGFA8X6PeKq06qEbZpbmXRs0Yfrcwn1hRONAQ5PLundgnI/oHxTIQUmj490jpXrKszvcgmynkrl
FHjHvtfp4ViVOtfmt1byubG9BaFRJe+L3+l7MkAodYyrC93/jisk3xi/veAkuRFa4F7qUioBOuRYXE
KSg4eF+tMouqbzKoM2O9vsAHrBRaIhV+yTiiDjN2UswzmQl4n4m/wRZ/OKISiewUzoBhf07431dutL
i3Lpl5IdaNMiYdsi9D8Mgb7N2x4DKZKTXOVnHmMN79yL1u2WUlp3vhWAmz8Af4Sux7jh
\end{tiny}
\newline\newline

\chapter{netflow}
Set up a netflow receiver on port 9996 and make a traffic billing.
\newline
For each user you should write bytes count and make it available via http://<your-ip>/billing.html which looks like:
\newline
\begin{minted}[fontsize=\scriptsize, samepage=true]{text}
<tr>
  <td>IP</td>
  <td>bytes count</td>
</tr>
\end{minted}
Update period: 1 minute
\newline
Sort: by bytes DESC
\newline\newline
Ставим flow-tools. Настраиваем. Теряем 3 часа, чтобы понять, что нам нужны только 220.123.31.0/24. Готово.
\begin{listing}[H]
  \inputminted[linenos=true, fontsize=\scriptsize, samepage=true]{text}{flow-capture.conf}
  \caption{flow-capture.conf}
  \label{lst:flow-capture.conf}
\end{listing}
\begin{listing}[H]
  \inputminted[linenos=true, fontsize=\scriptsize, samepage=true]{text}{report.conf}
  \caption{report.conf}
  \label{lst:report.conf}
\end{listing}
\begin{listing}[H]
  \inputminted[linenos=true, fontsize=\scriptsize, samepage=true]{bash}{flowrep.sh}
  \caption{flowrep.sh}
  \label{lst:flowrep.sh}
\end{listing}

\chapter{Repo}
We got a repositary at /root/repo, but it doesn't work with youm < 3.0.0. Fix it and make aviable via http://<ip>/repo.
\newline\newline
Сразу смотрим в ман по createrepo:
\newline
The older default was "sha", which is actually "sha1", however explicitly using "sha1" doesn't work on older (3.0.x) versions of yum, you need to specify "sha".
\newline
На всякий случай бэкапим. С дуру попробовал пересоздать repodata с указанием алгоритма sha, сломал. Восстановил, открыл руками repomd.xml и поменял тип и хэш-суммы.
\newline
\begin{minted}[fontsize=\scriptsize, samepage=true]{bash}
cd /root/repo/repodata/

for i in *.bz2; do
  sha_old=$(sha256sum $i);
  sha_new=$(shasum $i);
  sed -i -e "s/$sha_old/$sha_new/g" repomd.xml;
done

for i in *.bz2; do
  bzcat $i > ${i%.bz2};
  sha_old=$(sha256sum $i);
  sha_new=$(shasum $i);
  sed -i -e "s/$sha_old/$sha_new/g" repomd.xml;
  rm -f ${i%.bz2};
done

sed -i -e "s/\"sha256\"/\"sha\"/g" repomd.xml
\end{minted}

\chapter{Infected binary}
Your image has been touched by a cracker who replaced one of standard system binaries with his own. We thought that the program contains some secret string and it will output it when properly executed. Please find that string.
\newline\newline
Ну это совсем просто. Ищем изменённые бинарники (проверяем чексумму), затем смотрим в него.
\newline
\begin{minted}[fontsize=\scriptsize, samepage=true]{bash}
apt-get install debsums -y
debsums -a -s
strings /usr/bin/\[ | more
\end{minted}
Ага. Записать строчку в binary.txt и расшарить по http. Готово.
\newline\newline
Сделано быстрее всех %)

\chapter{DNS MITM}
Let's suppose you're an administrator of large corporate network. You have a list of hosts which should be blocked according to the company policy. You decide to set up your DNS server in such a way that it acts as normal DNS server for all the hosts except the ones from this list. For those hosts, it returns the address of itself to all A queries. Later you plan to set up a special web server that displays the page about your company Internet restrictions on the same machine.
\newline
Now the task is to set up such DNS server. You should find the list of hosts in /root.
\newline\newline
Сначала я подумал, что подменять запросы нужно будет для отдельных клиентских IP, и приготовился сделать 2 view в bind
\newline
Но оказалось, что блокировать нужно отдельные домены. ещё проще
\newline
\begin{minted}[fontsize=\scriptsize]{text}
# устанавливаем BIND и утилиту dig
apt-get install bind9 dnsutils

# создаём часть конфига, добавляем в него блокируемые домены
# некоторые домены повторяются, поэтому uniq
cat /root/dns.hosts | sort | uniq  | while read line; do
  echo "zone \"$line\" {";
  echo "    type master;";
  echo "    file \"/etc/bind/db.fake\";";
  echo "};";
done > /etc/bind/named.fake.conf

# часть результата:
root@postel:/etc/bind# head -n 8 /etc/bind/named.fake.conf
zone "12zeed.com" {
    type master;
    file "/etc/bind/db.fake";
};
zone "1ink.com" {
    type master;
    file "/etc/bind/db.fake";
};


# включаем его в основной конфиг /etc/bind/named.conf
include "/etc/bind/named.fake.conf";

# создаём файл зоны. один на всех, благодаря шаблону "@"
$TTL 86400
@       IN      SOA     postel.yandex.com.      postel.yandex.com (
                        2015042201
                        3H
                        15M
                        1W
                        1D )

@       IN      NS      postel
postel  IN      A       10.0.1.51
@       IN      A       10.0.1.51
*       IN      CNAME   @

# в другом конфиге/etc/bind/named.conf.options  включаем forwarders ( не обязательно, у нас уже есть есть zone "." type hint )
# и полное логирование ( а это вообще нежелательно, т.к. снижает производительность на порядки )

# создаём каталог для логов
mkdir /var/log/bind
chown -R bind /var/log/bind

# запускаем bind
service bind start

#проверяем:
root@postel:/etc/bind# dig +short ya.ru @localhost
213.180.193.3
213.180.204.3
93.158.134.3
root@postel:/etc/bind# dig +short $(head -n1 /root/dns.hosts) @localhost
10.0.1.51
\end{minted}
\begin{listing}[H]
  \inputminted[linenos=true, fontsize=\scriptsize, samepage=true]{text}{named.conf.options}
  \caption{named.conf.options}
  \label{lst:named.conf.options}
\end{listing}
P.S. когда описывал решение, обратил внимание что это решение не универсальное и даже не полностью выполняет задание
\newline
Что ж, баги проверяющего робота - это приятно ;)

\chapter{SVN}
You have svn repository in /root/repo. Delete (like svn rm) all files which are greater than 5MB in all revisions and make them available via svn://ip/
\newline
Big file should be deleted only in the revision when it became >5MB.
\newline\newline
К сожалению, мега-супер-пупер-крутой скрипт был утерян ввиду анализа Infected binary через gdb, так что вкратце алгоритм:
\begin{enumerate}
  \item Ставим более свежие версии svn и svn-utils из wheezy-backports
  \item Запускаем svnserve на /root/repo
  \item Делаем локальный клон через svn co
  \item Дальше скрипт:
  \begin{itemize}
    \item Дампить все ревизии через svn dump (1 - полная, остальные с --incremental)
    \item Смотреть какой файл превышает 5МиБ через svn ls --verbose
    \item Добавлять к списку exclude
    \item Прогонять через svndumpfilter
  \end{itemize}
  \item Получили 101 отфильтрованную ревизию
  \item Теперь через svnadmin create создаём репозиторий и через svn load грузим фильтрованный дамп
  \item Сервим новый реп
  \item PROFIT!
\end{enumerate}

\chapter{NFS}
Make nfs://10.10.10.11/dir available as http://<yourip>/nfs. Use this credentials user@YA.ROOT:password. NFS server's host name is localhost.
\newline\newline
Домен в логине - явный намёк на realm Kerberos'а
\newline
Где искать KDC ? Неизвестно. Предполагаем, что на том же 10.10.10.11
\newline
\begin{minted}[fontsize=\scriptsize]{text}
# ставим утилиты kerberos и nfs клиента
apt-get install krb5-user nfs-common

# добавляем в конфиг
[libdefaults]
default_realm = YA.ROOT
allow_weak_crypto = true

[realms]
YA.ROOT = {
  kdc = 10.10.10.11
  admin_server = 10.10.10.11
  default_domain = ya.root
}

[domain_realm]
.ya.root = YA.ROOT
ya.root = YA.ROOT

#проверяем
kinit -p user@YA.ROOT

#успешно получаем TGT, можно даже посмотреть его
klist

# Дальше был затык н несколько часов. Я использую nfs4 + kerberos дома уже лет 8 и глаз замылился.
# Всегда использовал sec=krb5i, а для этого нужен keytab на имя host/$client.$domain@$REALM.
# Keytab искали по обеим ВМ
# Пытался подобрать principal для keytab'а
# Даже зашёл на kadmin от имени того же user@YA.ROOT и пробовал искать и экспортировать principal'ы.
# Почему-то яндекс забыл закрыть доступ :)
# От безысходности экспортировал в keytab сам user@YA.ROOT
# Не прошло
# Зато оказалось, что при экспорте _меняется_пароль_. Ну ладно, kinit и kadmin можно выполнять и через keytab.
# Но т.к. . сервер был _общий_ на все команды, то я стал хакиром, который портил жизнь соперникам }:->
# Парни, я не специально. В запарке не сразу сообразил что сделал.
# Яндекс тоже опомнился не сразу, но чуть позже доступ к kadmin заблокировал
# Пробовал монтировать и без кейтаба, с ожидаемым эффектом
# mount -t nfs4 10.10.10.11:/dir /mnt/nfs -o sec=krb5i

# Потом larrabee ВНЕЗАПНО сказал, что шара смонтировалась
# монтировать нужно было без sec=krb5i, т.е. без проверки сервера и клиента через KDC. Ну что сказать, глаз замылился
mount -t nfs4 10.10.10.11:/dir /mnt/nfs

# После этого каталог был срочно экспортирован через nginx, пользотель nginx тоже получил TGT

sudo -u www-data /bin/bash
$ kinit -p user@YA.ROOT

# /etc/nginx/sites-enabled/default:

server {
  listen 80;
  server_name 10.0.1.51 localhost postel postel.yandex.com postel.ya.root;

  location /nfs {
    alias /mnt/nfs;
  }
  access_log /var/log/nginx/postel_access.log;
  error_log /var/log/nginx/postel_error.log;
}

#можно проверить wget'ом
wget http://0.0.0.0/nfs/$(ls -1 -f /mnt/nfs | head -n1 )
\end{minted}

\chapter{Nginx Lua}
We have inherited from previous admin several modules for nginx, you can find it out at /etc/nginx/lua.
\newline
No documentation, no examples of configuration files. But we have some examples, how web-server on port 8000 must work:
\begin{itemize}
  \item /static/local/jquery.min.js should return content of file /var/www/static/jquery.min.js
  \item /static/local/<size>/dog.png should return thumbnail of image /var/www/static/dog.png, with width and height limit equals to X. <size> is a verbal description of size ("small", "medium" and so on, full list is unknown). We have values for X for different sizes: 50, 100, 500, 1000, 2000.
  \item All requests to /static/local/* except requests for css- and js- files, should return error 403 unless user has a special authrization cookie
  \item Request to /auth/local/jquery.min.js should set authorization cookie with name "auth\_local/jquery.min.js", which is accepted by /static/local/jquery.min.js
\end{itemize}
This list is not complete! But it's all we have.
\newline  
Don't forget to setup caching. It is said these modules support them.
\newline\newline
\begin{listing}[H]
  \inputminted[linenos=true, fontsize=\scriptsize, samepage=true]{python}{nginx_lua.py}
  \caption{nginx\_lua.py}
  \label{lst:nginx_lua.py}
\end{listing}

\part{Итоги}
6 место. Огромное спасибо tazhate, madrouter, realloc, lumi, octocat, delirium, anonymous\_sama, trofk, feofan, tailgunner, всем кого забыл, ну и мне \%)
\newline\newline
\small 
\begin{tabular}{ | l | l | l | l | l | l | l | l | l | l | }
  \hline
  Backup & HTTPS MITM & CI & netflow & Repo & Infected binary & DNS MITM & SVN & NFS & Nginx Lua\\ \hline
  01:29:18 & 04:27:47 & 07:01:54 & 05:25:27 & 01:39:15 & 00:18:49 & 01:27:47 & 03:29:41 & 04:52:19 & 07:06:34\\
  \hline
\end{tabular}
\end{document}
